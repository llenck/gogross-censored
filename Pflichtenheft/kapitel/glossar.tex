\begin{table}[h]
	\centering
	\begin{tabularx}{\textwidth}{X X}
		\rowcolor[HTML]{C0C0C0}
		\textbf{Abkürzung} & \textbf{Beschreibung} \\
		SuS & Schüler und Schülerinnen. \\
		\rowcolor[HTML]{E7E7E7}
		Deeplink & Verweist im Gegensatz zu ,,Surface Links'' auf Unterseiten einer Website. Diese sind ohne Deeplink nicht erreichbar, d.h. werden auch nicht von Suchmachinen gefunden. \\
		Benutzermaske & Auch Bildschirmmaske, ist ein Begriff für die Benutzeroberfläche, womit eine wenig technisch-affine Person mit der Software interagieren kann.\\
		\rowcolor[HTML]{E7E7E7}
		Weboberfläche & Wie eine Benutzermaske, nur in einem Browser ausführbar. \\
		Mockup & Eine nichtfunktionale Version eines Benutzermasken-Designs. \\
		\rowcolor[HTML]{E7E7E7}
		Befehlszeile & Eine Befehlszeile ist das Eingabefeld in einem Terminal, einem Programm, dass exakt beschriebene Befehle, zumeist in einem Ordner, ausführt\\
		Server Command line & Eine Befehlszeile auf dem Server, auf die Software installiert ist und läuft. \\
		\rowcolor[HTML]{E7E7E7}
		JSON & Ein Machinenlesbares Datenformat. \\
		PK & Primary key: Identifizierende Kennung einer Zeile in einer Tabelle im relationalen Datenbank Modell. \\
		\rowcolor[HTML]{E7E7E7}
		FK & Foreign keys: Wert ist gleich einem PK in einer anderen Tabelle. \\
	\end{tabularx}
	\caption{Glossar: Der ersten Erwähnung nach sortiert}
	\label{table:glossar}
\end{table}
