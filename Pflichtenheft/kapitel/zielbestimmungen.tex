\section{Muss-Kriterien}
\begin{enumerate}
    \item Pro Planspiel können mehrere Großhändler von Lehrkräften angelegt werden, diese sind nach Ländern kategorisiert.
    \item Spielleiter:innen können alle Planspiele verwalten und haben Zugriff auf alle Daten, inklusive der Bestellungen.
    \item Lehrkräfte haben Zugriff auf alle Großhändler des eigenen Landes, und können Produkte, Kategorien sowie Verkäufe verwalten.
    \item Jedes SuS-Unternehmen kann sich einen Account pro Großhändler des eigenen Landes erstellen.
    \item Die Webshops der Großhändler haben eine Bestellfunktion (kompletter Checkout, mit Rechnungsstellung). Die einzige Bezahl-Methode ist „Rechnungsstellung mit Überweisung“.
    \item Der Zugriff auf den Webshop für die Kunden (SuS-Unternehmen) soll nur nach einer Anmeldung mit Benutzername und Kennwort möglich sein.
    \item Die Benutzermasken der Software sind englischsprachig.
    \item Zusätzlich zur Webseite gibt es eine Android-App, mit der SuS-Unternehmen die gleichen Funktionen wie auf der Webseite nutzen können.
\end{enumerate}

\section{Soll-Kriterien}
\begin{enumerate}
    \item Wenn sich ein SuS-Unternehmen registriert, bekommt es abhängig von dem Land nur die landinternen Großhändler angezeigt.
    \item Rechnungen sollen nach einem Kauf automatisiert an die E-Mail-Adresse des SuS-Unternehmens geschickt werden.
    \item Der Import / Export ganzer Großhändler sowie einzelner Produkte ist möglich.
    \item Preisnachlässe (Skonto) werden automatisch in den Endpreis miteinberechnet.
    \item Für die Account-Erstellung soll es einen Deeplink für Lehrer und einen Deeplink für Schüler geben.
    \item Die Software muss über eine intuitive Weboberfläche anwenderfreundlich bedienbar sein und schnell erlernbar sein.
    \item Nutzer können Passwörter über E-Mail zurücksetzen.
\end{enumerate}

\section{Kann-Kriterien}
\begin{enumerate}
    \item Gesamte Planspiele können importiert / exportiert werden (nur Spielleiter).
    \item Die Benutzermasken der Software sind multilingual (Deutsch, Dänsich, Niederländisch, Norwegisch).
    \item Es können mehrere Planspiele parallel laufen.
    \item Die Software soll modular aufgebaut sein, sodass spätere Erweiterungen möglich sind.
    \item Lehrkräfte können zeitbegrenzte Rabatte auf einzelnen Produkten einstellen.
\end{enumerate}

\section{Abgrenzungskriterien}
\begin{enumerate}
    \item Es gibt keine Mehrwertsteuer oder Umsatzsteuer.
    \item Es wird nur in Euro gehandelt.
    \item Registrierungsseiten sind nur über Deeplinks aufrufbar.
\end{enumerate}
